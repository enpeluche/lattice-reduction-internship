\section{Comprendre le rôle du shift dans le cas euclidien}

On rappel que calculer le degré de ligne décalé d'une matrice revient à calculer le degré de ligne de cette matrice multipliée à gauche par une matrice diagonale.

On pourrait ici définir une notion de décalage avec une matrice ligne-orthogonale. Cela a donné lieu à l'algorithme suivant que j'ai écris et dont j'ai prouvé la correction par moi-même.

\begin{smallalgo}{\textsc{shiftLLL}}{algo:shiftLLL}
    \KwIn{Une base \( G \) de \( \LL \), \( S^* \) une matrice ligne-orthogonale}
    \KwOut{Une base \( B \) de \( \LL \) tel que \( BS^* \) soit \( \mathrm{LLL} \)-réduite}

    \KwRet{ \( \mathrm{LLL}(G S^*) \cdot (S^*)^{-1} \) }
\end{smallalgo}


\begin{theoreme}
    L'algorithme \hyperref[algo:shiftLLL]{\emph{shiftLLL}} calcule correctement une base \( B \) du réseau \( \LL \) telle que \( BS^* \) soit \( \mathrm{LLL} \)-réduite.
\end{theoreme}

\begin{proof}
    On a le diagramme commutatif suivant en voyant \( S^* \) comme la matrice de passage de la base canonique \( \mathcal{C} \) à la base \( \mathcal{C'} \coloneqq S^* \). L'écriture \( \LL_{\mathcal{C}} \) où \( \mathcal{C} \) est une base représente \( \LL \) exprimée dans la base \( \mathcal{C} \).

    \[\begin{tikzcd}
            {\LL_{\mathcal{C}}} && {\LL_{\mathcal{C}}} \\
            \\
            && {\LL_{\mathcal{C}'}}
            \arrow["G", from=1-1, to=1-3]
            \arrow["{GS^*}"', from=1-1, to=3-3]
            \arrow["{S^*}", from=1-3, to=3-3]
        \end{tikzcd}\]

    Comme \( LLL(GS^*) \) est une base de \( \LL_{\mathcal{C}'}\), on en déduit le diagramme commutatif suivant et donc que \( LLL(GS^*) (S^*)^{-1} \) est une base de \( \LL_{\mathcal{C}} \).

    \[\begin{tikzcd}
            {\LL_{\mathcal{C}}} && {\LL_{\mathcal{C}}} \\
            \\
            && {\LL_{\mathcal{C}'}}
            \arrow["{\mathrm{LLL}(GS^*)(S^*)^{-1}}", from=1-1, to=1-3]
            \arrow["{\mathrm{LLL}(GS^*)}"', from=1-1, to=3-3]
            \arrow["{S^*}", from=1-3, to=3-3]
        \end{tikzcd}\]

    On a par construction de \( \mathrm{LLL} \) que \( LLL(GS^*) (S^*)^{-1} S^*\) est \( \mathrm{LLL}\)-réduite.
\end{proof}

L'algorithme termine car \( \mathrm{LLL}\) termine et sa complexité est de l'ordre de la complexité de \( \mathrm{LLL}\).

En revenant à la décomposition précédente, en calculant une base telle que \(V_1 V_2^*\) est \( \mathrm{LLL}\)-réduite, c'est-à-dire \(V_1 \) est \( \mathrm{LLL}\)-réduite pour le décalage \(V_2^*\). On peut se poser les questions suivantes qui sont des pistes à explorer :

Est-ce que \( (V_2^*)^{-1} U_2 V_2 \) est "quasi" \( \mathrm{LLL}\)-réduite ?

\begin{problem}[Hypothèse]
\( (V_2^*)^{-1} U_2 V_2 \) est une base propre, c'est-à-dire qu'en écrivant sa décomposition de Gram-Schmidt, les coefficients correspondant ne sont pas trop grand.
\end{problem}

La condition de Lovász peut être interprétée comme une forme de \(2\)-quasi-croissance.
Une question naturelle serait alors de se demander si, par analogie, le produit calculé par LLL pourrait satisfaire une propriété de \(4\)-quasi-croissance.