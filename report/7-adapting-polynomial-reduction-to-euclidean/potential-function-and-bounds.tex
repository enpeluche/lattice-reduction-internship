
\section{Fonction potentielle et bornes}

\subsection{Les cas polynomial}

Dans l’algorithme \hyperref[algo:WeakPopovForm]{\emph{\parencite{Mulders2003}}}, chaque opération entraîne soit une diminution du degré total par lignes, soit un déplacement du pivot d’une ligne vers la gauche.

\parencite{nielsen2013gmssr} définit une fonction de \emph{valeur} pour les vecteurs :
\[
    \psi :
    \begin{array}{rcl}
        \mathbb{F}[x]^{\ell+1} & \longrightarrow & \mathbb{N}_0                                                \\
        \mathbf{v}             & \longmapsto     & (\ell + 1) \cdot \rdeg \mathbf{v} + \mathrm{LP}(\mathbf{v})
    \end{array}
\]

où \( \mathrm{LP}(\mathbf{v} \) désigne l'indice du pivot de \( \mathbf{v} \).

\begin{lemma}
    Soit \( \mathbf{v}_j' \) le vecteur qui remplace \( \mathbf{v}_j \) lors d’une réduction de ligne dans \hyperref[algo:WeakPopovForm]{\emph{\parencite{Mulders2003}}}.
    Alors :
    \[
        \psi(\mathbf{v}_j') < \psi(\mathbf{v}_j).
    \]
\end{lemma}

Ainsi, l’algorithme \hyperref[algo:WeakPopovForm]{\emph{\parencite{Mulders2003}}} repose sur une fonction potentielle qui décroît strictement à chaque opération. Cette fonction atteint nécessairement une valeur minimale, correspondant à une situation où aucune opération supplémentaire n’est possible, c’est-à-dire lorsque la matrice est réduite en ligne. Cette même idée se traduit dans l'algorithme.

\subsection{Le cas entier}

Le lecteur pourra utilement se référer à la démonstration de la terminaison de l’algorithme \( \mathrm{LLL} \). Dans cette preuve, une quantité \( D \) a été introduite, qui décroît d’un facteur \( \frac{3}{4}\) à chaque échange de vecteurs. Cette décroissance strictement contrôlée constitue l’invariant principal garantissant la terminaison de l’algorithme.

\begin{definition}[Rappel]
    On définit \( \displaystyle D \coloneqq \prod_{1 \leq k < n} d_k = \prod_{1 \leq k < n} \| \bb_k^*\|^{2(n-k)}\)
\end{definition}

En pratique, la valeur de \( D \) peut devenir très grande, il est donc plus pertinent de regarder l’ordre de grandeur en considérant son logarithme \( \log(D) \), c’est-à-dire son nombre de bits, puisque c'est comme ça qu'il est considérer dans la démonstration.

\begin{proposition}
    \( \displaystyle \log(D) = \sum_{k=1}^{n-1} 2(n-k) \log (\| \bb_k^*\|)\)
\end{proposition}

On impose dans la preuve de terminaison que \( D \geq 1 \). Toutefois, \( D = 1 \) si et seulement si le réseau est isomorphe à \( \mathbb{Z}^n \). Un raisonnement similaire montre que très peu de réseaux satisfont \( D = 2 \). En pratique, la valeur de \( D \) reste largement supérieure à \( 1 \), et l’algorithme retourne une base \( \mathrm{LLL} \)-réduite avec \( D \gg 1 \). Ainsi, bien que le critère de terminaison repose sur une décroissance stricte de \( D \), cela ne reflète pas toujours le fait que l’algorithme a effectivement atteint un état suffisant de réduction.

On peut essayer d'améliorer la borne inférieure, on se place dans l'hypothèse que la base est \( \mathrm{LLL} \)-réduite et donc satisfait la condition de Lovàsz.

\[
    \begin{aligned}
        D & = \prod_{1 \leq k < n} \| \bb_k^*\|^{2(n-k)}                                                                                     \\
          & = \|\bb_1^*\|^{2(n-1)} \x \|\bb_2^*\|^{2(n-2)} \x \cdots \x \|b_n^*\|^2                                                          \\
          & \geq \|\bb_1^*\|^{2(n-1)} \x \left(\frac{\|\bb_1^*\|}{2}\right)^{2(n-2)} \x \cdots \x \left(\frac{\|\bb_1^*\|}{2^{n-2}}\right)^2 \\
          & \geq \left( \frac{\|b_1^*\|}{2^{\frac{4}{3} (n-2)}} \right)^{(n-1)n}
    \end{aligned}
\]

Après la remise de ce rapport, mon travail consistera à explorer la portée pratique de cette borne. Contrairement au cas polynomial, le comportement de \( D \) dans le cas entier reste difficile à cerner, et l’on ne sait pas précisément jusqu’où cette quantité peut décroître.

De manière similaire au raisonnement précédent, une borne supérieure sur \( D \) est également disponible.
Dans \parencite{MCA}, on trouve :
\[
    D \leq \left( \max_{1 \leq i \leq n} \| \bg_i \| \right)^{n(n-1)}.
\]
Une question naturelle est alors de savoir si cette borne peut être améliorée. On peut voir que l'exposant \( n(n-1) \) peut devenir \( (n-1)n \). Plutôt que d'utiliser \( \max_{1 \leq i \leq n} \| \bg_i \| \), on pourrait utiliser un invariant lié au réseau comme \( |\LL| \).