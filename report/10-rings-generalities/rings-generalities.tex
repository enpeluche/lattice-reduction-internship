\chapter{Rappels sur les anneaux}

\label{appendix:ring}

\begin{tikzpicture}[remember picture, overlay]
	\node[anchor=north east, xshift=-2.5cm, yshift=-5cm]
	at (current page.north east)
	{
		\includegraphics[width=3cm]{images/logo/ring.jpeg}
	};
\end{tikzpicture}

\section{Généralités}
Ces résultats sont classiques de la théorie des anneaux commutatifs. Pour plus de détails, on pourra se reporter à un manuel standard d’algèbre.

\begin{definition}
	Un \textbf{anneau} $(A, +, \cdot)$ est un ensemble $A$ muni de deux lois internes :
	\begin{itemize}
		\item $(A, +)$ est un \textbf{groupe abélien} (on note $0$ son élément neutre et $-a$ l’inverse de $a$),
		\item la multiplication $\cdot : A \x A \to A$, notée simplement $ab$, est \textbf{associative} et \textbf{distributive} par rapport à $+$, c’est-à-dire :
		      \[
			      a(b + c) = ab + ac,
			      \quad (b + c)a = ba + ca,
			      \quad \forall a, b, c \in A,
		      \]
		      et possède un \textbf{élément neutre} $1 \in A$ (on dit alors que $A$ est un anneau \textbf{unitaire}).
	\end{itemize}
\end{definition}

\begin{definition}
	Un anneau $A$ est \textbf{commutatif} si $ab = ba$ pour tout $a, b \in A$.
\end{definition}

Dans ce mémoire et dans cette annexe, tous les anneaux seront supposés commutatifs par défaut, conformément aux usages standards, sauf indication explicite du contraire.

\begin{definition}
	Un anneau $A$ est \textbf{intègre} si, pour tout $a, b \in A$, on a
	\[ab = 0 \quad\Longrightarrow\quad (a=0 \ \text{ou}\ b=0).\]
\end{definition}
\begin{remark}
	Autrement dit, $A$ est intègre si et seulement si il n'a pas de diviseurs de $0$.
\end{remark}

\begin{definition}
	Soit $A$ un anneau commutatif. Un \textbf{idéal} $I \subseteq A$ est un sous-groupe additif de $(A, +)$ tel que, pour tout $a \in A$ et tout $x \in I$, on ait
	\(ax \in I.\)
\end{definition}

\begin{remark}
	En commutatif, $ax = xa$, donc une seule condition suffit à le décrire.
\end{remark}

\begin{definition}
	Un idéal $I$ de $A$ est dit \textbf{principal} s’il existe un unique $r \in A$ tel que $I = \langle r \rangle = \{ar : a \in A\}$. Un anneau $A$ est \textbf{principal} si tous ses idéaux sont principaux.
\end{definition}

\begin{definition}
	Un anneau $A$ est \textbf{noethérien} si toutes ses chaînes d’idéaux (i.e. $I_1 \subseteq I_2 \subseteq \ldots$) sont stationnaires, c’est-à-dire qu’il n’existe pas de chaîne infinie strictement croissante d’idéaux.
\end{definition}

\begin{remark}
	Cette définition est équivalente à dire que tout idéal de $A$ est de type fini, i.e. $I = \langle x_1,\dots,x_k\rangle$ pour un nombre fini d’éléments.
\end{remark}

\begin{definition}[Anneau factoriel]
	Un anneau intègre $A$ est \textbf{factoriel} si tout élément non nul et non inversible de $A$ admet une factorisation en éléments irréductibles unique à l’ordre près (et inversibles près).
\end{definition}

Tout \textbf{anneau euclidien} est principal, tout \textbf{anneau principal} est factoriel et tout \textbf{anneau principal} est noethérien.

\begin{table}[h]
	\centering
	\caption{Quelques exemples d'anneaux et leurs propriétés}
	\begin{tabular}{|c|c|c|c|c|c|}
		\hline
		\textbf{Anneau}        & \textbf{Commutatif} & \textbf{Intègre} & \textbf{Principal} & \textbf{Factoriel} & \textbf{Noethérien} \\
		\hline
		$\Z$                   & \checkmark          & \checkmark       & \checkmark         & \checkmark         & \checkmark          \\
		$\Z/n\Z$ (non premier) & \checkmark          & \xmark           & \checkmark         & \xmark             & \checkmark          \\
		$\Z/p\Z$ ($p$ premier) & \checkmark          & \checkmark       & \checkmark         & \checkmark         & \checkmark          \\
		$\Z[i]$                & \checkmark          & \checkmark       & \checkmark         & \checkmark         & \checkmark          \\
		$\Z[\sqrt{-5}]$        & \checkmark          & \checkmark       & \xmark             & \xmark             & \checkmark          \\
		$\Z \x \Z$             & \checkmark          & \xmark           & \xmark             & \xmark             & \checkmark          \\
		$M_n(\K)$, $n \geq 2$  & \xmark              & \xmark           & —                  & —                  & \checkmark          \\
		$C^0([0,1], \R)$       & \checkmark          & \checkmark       & \xmark             & \xmark             & \xmark              \\
		\hline
	\end{tabular}
\end{table}

\graphicspath{ {10-rings-generalities/images/} }

\section{Anneaux de polynômes}

Les matrices polynomiales que nous étudierons dans ce mémoire sont construites à partir d’anneaux de polynômes, qui en forment la base algébrique.

\begin{theoreme}
    Si $A$ est un anneau factoriel, alors $A[x]$ est aussi factoriel.

    Si $A$ est un anneau noethérien, alors $A[x]$ est noethérien.

    Si $A$ est intègre, alors $A[x]$ est intègre.
\end{theoreme}

\begin{proposition}
    Soit $\K$ un corps. Alors l’anneau de polynômes $\K[x]$ est \textbf{principal} et l’anneau $\K[x,y]$ \textbf{n’est pas} principal.
\end{proposition}

\begin{remark}
    Quand on a deux variables, la structure d’idéaux se complique et ne peut pas être engendrée par un seul polynôme dans la plupart des cas.
\end{remark}

\begin{table}[h]
    \centering
    \caption{Exemples d’anneaux de polynômes et leurs propriétés}
    \begin{tabular}{|c|c|c|c|c|c|}
        \hline
        \textbf{Anneau}           & \textbf{Commutatif} & \textbf{Intègre} & \textbf{Principal} & \textbf{Factoriel} & \textbf{Noethérien} \\
        \hline
        $\K[x]$                   & \checkmark          & \checkmark       & \checkmark         & \checkmark         & \checkmark          \\
        $\K[x,y]$                 & \checkmark          & \checkmark       & \xmark             & \checkmark         & \checkmark          \\
        $\Z[x]$                   & \checkmark          & \checkmark       & \xmark             & \checkmark         & \checkmark          \\
        $\F_p[x]$                 & \checkmark          & \checkmark       & \checkmark         & \checkmark         & \checkmark          \\
        $\F_p[x,y]$               & \checkmark          & \checkmark       & \xmark             & \checkmark         & \checkmark          \\
        $\R[x]/(x^2 + 1)$         & \checkmark          & \checkmark       & \checkmark         & \checkmark         & \checkmark          \\
        $\K[x]/(x^n)$, $n \geq 2$ & \checkmark          & \xmark           & \checkmark         & \xmark             & \checkmark          \\
        \hline
    \end{tabular}
\end{table}