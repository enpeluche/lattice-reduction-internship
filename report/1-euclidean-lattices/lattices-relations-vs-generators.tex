
\section{Réseaux définis par relations plutôt que par générateurs}

\begin{definition}
    Soit $\LL \subset \R^n$ un réseau de base $(\bb_i)_{1 \leq i \leq n}$. Le \textbf{dual} de \( \LL \) est défini par
    \[
        \LL^\vee:=\{ \bx \in \R^n ~|~ \forall \by \in \LL, \langle \bx, \by \rangle \in \Z\}.
    \]
\end{definition}

\begin{proposition}
    On a $\LL = \LL(\bb_1, \ldots, \bb_m)$ si et seulement si $\LL^\vee= \LL(\bb_1^\vee, \ldots, \bb_n^\vee)$,

    où \( \bb_i^\vee \) vérifie \( \left \langle \bb_i^\vee, \bb_i \right \rangle = \delta_{i,j}\). \footnote{ \( \delta_{i,j} \) est le symbole de Kronecker, il vaut \( 1 \) si \( i = j \) et \( 0 \) sinon.}
\end{proposition}

\begin{proposition}
    $\LL^\vee$ est un réseau de base :

    \begin{itemize}
        \item [$\bullet$] $(B^t)^{-1}$ lorsque le réseau est de rang plein.
        \item[$\bullet$] $ B(B^t B)^{-1} $ lorsque le réseau n'est pas de rang plein.
    \end{itemize}
\end{proposition}

\begin{proposition} De la proposition précédente découle les propriétés suivantes :
    \begin{itemize}
        \item[$\bullet$] $\rang(\LL) = \rang(\LL^\vee)$.
        \item[$\bullet$] $|\LL^\vee| = |\LL|^{-1}$.
        \item[$\bullet$] \( (\LL ^\vee)^\vee = \LL \)
    \end{itemize}
\end{proposition}

\begin{definition}
    On dit que \( \LL \) est \textbf{auto-dual} si \( \LL = \LL^\vee \).
\end{definition}

\begin{proposition} On a les propriétés suivantes :
    \begin{itemize}
        \item[$\bullet$] $(a \LL)^\vee = \frac{1}{a} \LL^\vee$ pour tout $a \in \R^*$.
        \item[$\bullet$] $(\Z u)^\vee= \frac{1}{ \|u\|} \Z u$ pour tout $u \in \R^m \backslash \{ \mathbf{0} \}$.
    \end{itemize}
\end{proposition}

\begin{example}
    Le réseau dual de $\Z^n$ est $\Z^n$ et est donc auto-dual.
\end{example}

\begin{figure}[h]
    \centering
    \begin{subfigure}[b]{0.25\textwidth}
        \centering
        \includegraphics[width=\textwidth]{images/lattice_0_5.png}
        \caption{\( 2 \Z^2 \)}
    \end{subfigure}
    \hspace{0.05\textwidth}
    \begin{subfigure}[b]{0.25\textwidth}
        \centering
        \includegraphics[width=\textwidth]{images/lattice_1_5.png}
        \caption{\( \frac12 \Z^2  \)}
    \end{subfigure}
    \caption{Un réseau et son dual.}
    \label{fig:dual}
\end{figure}


\begin{proposition}
    Soit $\LL_1$, $\LL_2$ des réseaux, alors
    $$(\LL_1 \oplus \LL_2)^\vee = \LL_1^\vee \oplus \LL_2^\vee$$
\end{proposition}

\begin{lemma}
    Soit $\LL$ un réseau de dimension $n$. On a
    \begin{itemize}
        \item[$\bullet$] $\lambda_1(\LL) \cdot \lambda_1(\LL^\vee) \leq n$,
        \item[$\bullet$] $\lambda_1(\LL) \cdot \lambda_n(\LL^\vee) \geq 1$.
    \end{itemize}
\end{lemma}

\parencite{Banaszczyk1993} a démontré une relation encore plus forte entre les minima d'un réseau et ceux de son dual, connue sous le nom de théorème de transfert.

\begin{theoreme}[Théorème de transfert]
    Soit $\LL$ un réseau de dimension $n$. On a
    \[
        1 \leq \lambda_1(\LL) \cdot \lambda_n(\LL^\vee) \leq n.
    \]
\end{theoreme}

\begin{definition}
    On définit le réseau euclidien \( A_n \subset \R^{n+1} \) par :
    \[A_n = \left\{ (x_1, \dots, x_{n+1}) \in \Z^{n+1} \;\middle|\; \sum_{i=1}^{n+1} x_i = 0 \right\}\]
\end{definition}

\begin{example}
    On a :
    \begin{align*}
        A_0 & = \{ 0 \} \subset \R,                                              \\
        A_1 & = \{ (x, -x) \in \Z^2 \}, \text{ donc } A_1 \text{ est de rang } 1 \\
        A_2 & = \{ (x, y, z) \in \Z^3 \mid x + y + z = 0 \},                     \\
            & = \langle a_1, a_2 \rangle \quad \text{où }
        a_1 = (1, -1, 0), \quad a_2 = (0, 1, -1),                                \\
            & \text{donc } A_2 \text{ est de rang } 2.
    \end{align*}
\end{example}

\begin{proposition}
    Pour tout $n \in \N$, \( A_n \) a pour matrice génératrice :

    \[
        B_n :=
        \begin{pmatrix}
            1      & 0      & 0      & \cdots & 0      \\
            -1     & 1      & 0      & \cdots & 0      \\
            0      & -1     & 1      & \cdots & 0      \\
            \vdots & \vdots & \ddots & \ddots & \vdots \\
            0      & 0      & \cdots & -1     & 1      \\
        \end{pmatrix}
        \in M_n(\Z)
    \]
\end{proposition}

\begin{proposition}
    \( A_n \) est un réseau euclidien de rang \( n \), pour tout \( n \in \N \).
\end{proposition}

Il existe une classe particulière de réseaux qui joue un rôle important en cryptographie.
\begin{definition}[Réseau $q$-aire]
    Un réseau $\LL$ est un réseau $q$-aire si
    \[
        q\mathbb{Z}^n \subseteq \LL \subseteq \mathbb{Z}^n.
    \]
\end{definition}


Étant donnée une matrice $\mathbf{A} \in \Z_q^{m \times n}$ pour certains entiers $n, m, q \in \N$, on peut définir deux réseaux :

\[
    \LL_q(\mathbf{A}) = \left\{ \by \in \Z^m : \by = \mathbf{A}\mathbf{s} \bmod q \text{ pour un certain } \mathbf{s} \in \Z^n \right\}
\]

\[
    \LL_q^\perp(\mathbf{A}^T) = \left\{ \by \in \Z^m : \mathbf{A}^T \by = 0 \bmod q \right\}
\]

Les deux réseaux sont de dimension $m$. Le premier est engendré par les lignes de $\mathbf{A}$ et a pour déterminant $q^{m-n}$, tandis que le second contient tous les vecteurs orthogonaux aux lignes de $\mathbf{A}$ et a pour déterminant $q^n$.

De plus, ils sont liés par la dualité des réseaux, c’est-à-dire :
\[
    \LL_q^\perp(\mathbf{A}^T) = q \cdot \LL_q(\mathbf{A})^\vee \quad \text{et} \quad \LL_q(\mathbf{A}) = q \cdot \LL_q^\perp(\mathbf{A}^T)^\vee.
\]

