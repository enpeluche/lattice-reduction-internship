\chapter*{Notations, complexité et acronymes}

\addcontentsline{toc}{chapter}{Notations, complexité et acronymes}

Cette section fait office de lexique et recense l'ensemble des notations et acronymes utilisées tout au long de ce mémoire.

\section*{\textit{Notations}}

\begin{tabular}{@{}p{3cm}p{12cm}@{}}
	\( \Z \)               & L'ensemble des entiers relatifs.                                                 \\
	\( \Z_n \)             & L'ensemble des entiers modulo \( n \).                                           \\
	\( \Q \)               & L'ensemble des rationnels.                                                       \\
	\( \R \)               & L'ensemble des réels.                                                            \\
	\( \K \)               & Un corps quelconque.                                                             \\
	\( \F_p \)             & Un corps fini de caractéristique \( p \).                                        \\
	\( \F_p[x] \)          & L'anneau des polynômes univariés à coefficients dans le corps fini \( \F_p \).   \\
	\( \F_p^{\leq d}[x] \) & Les polynômes à coefficients dans \( \F_p \) de degré inférieur ou égal \( d \). \\
	\( \F_p[x]^{m\x n} \)  & Les matrices \( m\x n \) à coefficients dans \( \F_p[x] \).                      \\
	\( \LL \)              & Un réseau, désigné par une lettre majuscule calligraphiée.                       \\
	\( \LL(B) \)           & Le réseau engendré par la matrice \( B \).                                       \\
	\( B^t \)              & La transposée de la matrice \( B \).                                             \\
	\( B^* \)              & La base de Gram-Schmidt associée à \( B \).                                      \\
	\( \bb, \bg, \ldots \) & Les vecteurs de \( \R^n \), notés en gras.
\end{tabular}

\section*{\textit{Complexité}}
\begin{tabular}{@{}p{6cm}p{12cm}@{}}
	Multiplication dans \( \F_p^{\leq d}[x] \)         & \( \M(d) = \OO(d \x \log d \x \log \log d) \).                                                \\
	Multiplication dans \( \F_p^{n\x n} \)             & \( \MM(n) = \OO(n^\omega) \).                                                                 \\
	Multiplication dans \( \F_p^{\leq d}[x]^{m\x n} \) & \( \MM(n,d)\footnotemark[1] = \OO(\MM(n) \M(d)) = \tilde{\OO}(n^\omega\footnotemark[2] d) \). \\
\end{tabular}\\

\footnotetext[1]{\( \MM(n,d) \) peut également être obtenu via une approche d'évaluation-interpolation sur une suite géométrique, ce qui permet d'améliorer certaines bornes de complexité.}
\footnotetext[2]{\( \omega \) est l'exposant optimal de la multiplication matricielle.}

\section*{\textit{Symboles}}
\begin{tabular}{@{}p{3cm}p{12cm}@{}}
	\( \qed \)                  & Le symbole marquant la fin d'une démonstration. \\
	\( \leftarrow, \coloneqq \) & Les symboles d'affectations d'un algorithme.
\end{tabular}\\


\section*{\textit{Acronymes}}

\begin{tabular}{@{}p{3cm}p{10cm}@{}}
	\(\mathrm{LLL}\)  & Lenstra–Lenstra–Lovász              \\
	\(\mathrm{BKZ}\)  & Block Korkine-Zolotarev             \\
	\(\mathrm{SVP}\)  & Shortest Vector Problem             \\
	\(\mathrm{SIVP}\) & Shortest Independant Vector Problem \\
	\(\mathrm{CVP}\)  & Closest Vector Problem
\end{tabular}