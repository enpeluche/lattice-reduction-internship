\section{Modules sur un anneau principal}

Dans la suite, on considère un anneau principal $A$, c’est-à-dire un anneau (commutatif unitaire) dans lequel \emph{tout idéal} est principal.

\begin{theoreme}
    Soit $M$ un module \textbf{libre} sur un anneau principal $A$. Alors tout sous-module de $M$ est également libre et son rang est inférieur ou égal à celui de $M$.
\end{theoreme}

\begin{example}
    Dans le $\Z$-module libre $\Z^2$, considérons le sous-ensemble
    \[
        N \;=\; \{(a,b)\in \Z^2 \mid a \equiv b \pmod{10}\}.
    \]
    On constate que $N$ est un \textbf{sous-module} de $\Z^2$, et qu’il est engendré par les vecteurs $(1,1)$ et $(0,10)$. Ceux-ci sont $A$-linéairement indépendants, donc $N$ est libre de rang $2$, tout comme $\Z^2$ lui-même.
\end{example}

\begin{table}[h!]
    \centering
    \renewcommand{\arraystretch}{1.2}
    \begin{tabular}{|c|c|c|c|c|c|}
        \hline
        \textbf{Module $M$}             & \textbf{Anneau $A$} & \textbf{Libre} & \textbf{Type fini} & \textbf{Torsion} & \textbf{Réf.} \\
        \hline
        $\K[x]^n$                       & $\K[x]$             & Oui            & Oui                & Non              & (1)           \\
        $\K[x]/(x^n)$                   & $\K[x]$             & Non            & Oui                & Oui              & (2)           \\
        $\K[x]^\infty$                  & $\K[x]$             & Oui            & Non                & Non              & (3)           \\
        $\K(x)$                         & $\K[x]$             & Non            & Non                & Oui              & (4)           \\
        $(\K[x])^n / (x \cdot \K[x])^n$ & $\K[x]$             & Non            & Oui                & Oui              & (5)           \\
        \hline
    \end{tabular}
    \caption{Exemples de modules sur l’anneau $\K[x]$}
\end{table}

\vspace{1em}
\noindent
\textbf{Commentaires sur les exemples :}
\begin{enumerate}
    \item \(\K[x]^n\) est le module libre canonique : c’est un \(\K[x]\)-module libre de rang \(n\). Il sert de modèle aux réseaux polynomiaux.
    \item \(\K[x]/(x^n)\) est un module de torsion : tout élément est annulé par une puissance de \(x\). Il n’admet pas de base libre.
    \item \(\K[x]^\infty\) (somme directe infinie) est un module libre, mais non de type fini. Il possède une base infinie indexée par \(\N\).
    \item \(\K(x)\), le corps des fractions rationnelles, est un module divisible mais non libre. Il contient des éléments sans expression unique comme combinaison de base.
    \item $(\K[x])^n / (x \cdot \K[x])^n$ est un module quotient, utilisé dans les algorithmes de bases d’ordre modulo $x^\sigma$. C’est un module de torsion.
\end{enumerate}
