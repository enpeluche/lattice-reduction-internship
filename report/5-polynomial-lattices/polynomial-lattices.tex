\graphicspath{ {5-polynomial-lattices/images/} }

\chapter{Réseaux polynomiaux}

\lettrine{L}{orsque} les coefficients des matrices appartiennent à un corps \( \K \), les opérations
classiques telles que la multiplication, l'inversion, le calcul du déterminant ou la résolution de
systèmes linéaires possèdent des complexités comparables. En revanche, lorsqu'on considère des matrices
à coefficients dans l'anneau \( \K[x] \), des différences apparaissent : si le calcul du déterminant
conserve la même complexité que celle du produit matriciel, l'inversion est plus coûteuse. Cette
complexité découle de la structure de l'anneau \( \K[x] \) : bien qu'il s'agisse d'un anneau principal
(à la différence de \( \K[x,y] \)), il ne s'agit pas d'un corps. Ainsi, certaines opérations, comme
l’inversion, ne sont plus systématiquement réalisables. Notamment, dans \( \K[x] \), seuls les polynômes
constants non nuls sont inversibles. Cette restriction impose de repenser et redéfinir rigoureusement
plusieurs notions de l'algèbre linéaire. Les matrices à coefficients dans \( \K[x] \) sont essentielles
dans de nombreuses applications. \footnote{Par exemple dans l’interpolation bivariée, une étape centrale
    du décodage des codes de Reed-Solomon}.  Ce chapitre vise à explorer les réseaux polynomiaux  et leurs
propriétés spécifiques.
Ce mémoire avait pour ambition initiale de motiver rigoureusement l’introduction des bases réduites dans
le cadre des matrices polynomiales. Toutefois, afin de ne pas aborder un sujet trop éloigné des objectifs
du stage, et par souci de concision, nous ne détaillerons pas ici les aspects liés aux mesures de complexité.
Ce que j'avais rédigé initialement se retrouve en annexe. Notons simplement qu’il est essentiel d’analyser
finement le comportement des matrices polynomiales vis-à-vis des opérations algébriques usuelles, en
particulier la multiplication. Cela justifie l’introduction de la notion de degré de ligne, qui jouera un
rôle central dans le chapitre suivant.

\input{5-polynomial-lattices/polynomial-lattices-definitions-examples}
\input{5-polynomial-lattices/notion-row-degree}