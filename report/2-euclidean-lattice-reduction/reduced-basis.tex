\section{La base réduite}

Dans les problèmes liés aux réseaux euclidiens, une base orthogonale représenterait une base idéale.

\begin{problem}[\textbf{Problème}]
La base \( B^* \) ( de $\R^n$ ) n'est généralement pas une base du réseau \( \LL(B) \).
\end{problem}

\begin{figure}[H]
    \centering
    \includegraphics[width=0.23\textwidth]{lattice_0_9.png}
    \caption{ Base de Gram-Schmidt qui n'est pas une base de \( \LL \). }
    \label{fig:GS}
\end{figure}

Nous cherchons une base de \( \LL \) qui \emph{approxime} la base de Gram--Schmidt aussi fidèlement que possible.

\begin{notation}
    On définit l'entier le plus proche de $x \in \R$ par $\lceil x\rfloor  = \lfloor x+1/2 \rfloor$.
\end{notation}

\begin{proposition}
    On a $|x-\lceil x\rfloor| \leq \frac12$ pour tout $x \in \R$.
\end{proposition}

Dans la suite de cette explication, le lecteur est invité à porter une attention particulière aux * qui dénotent un vecteur de la base de Gram-Schmidt.
L'opération d'orthogonaliation de Gram-Schmidt nous donne :
\begin{equation}
    \mathbf{b}^*_2 \coloneqq \mathbf{b}_2 - \mu_{2,1} \mathbf{b}_1^* \notin \LL
\end{equation}

Mais \( \mathbf{b}_1^* = \mathbf{b}_1 \in \LL \), en prenant \( k \in \Z \), on a
\(
\mathbf{b}_2 - k \mathbf{b}_1 \in \LL
\),
on pourrait donc réaliser l'opération
\[
    \mathbf{b}_2 \coloneqq \mathbf{b}_2 - k \mathbf{b}_1 \in \LL
\]
On peut donc choisir \( k \in \Z \) qui minimise \( \left \langle \bb_2, \bb_1 \right \rangle \), ce qui est réalisé par \( k \coloneqq \nint{\mu_{i,j}}\). On en déduit donc l'opération
\begin{equation}
    \mathbf{b}_2 \coloneqq  \mathbf{b}_2 - \nint{\mu_{2,1}} \mathbf{b}_1 \in \LL
\end{equation}

On peut donc étendre ce procédé par récurrence pour construire la nouvelle famille \( (\bb_i)_{1 \leq i \leq n} \).

En refaisant les calculs on peut montrer que la base de Gram-Schmidt associée est inchangée, mais que les nouveaux coefficients \( |\mu_{i,j}|<\frac12\) d'après la proposition 2.1.

\begin{figure}[H]
    \centering
    \includegraphics[width=0.23\textwidth]{lattice_1_9.png}
    \caption{ Base de \( \LL \) proche de la base Gram-Schmidt . }
    \label{fig:GS}
\end{figure}

\begin{definition}
    Soit \( (\bb_i)_{1 \leq i \leq n} \) une base d'un réseau et \( U \) la matrice triangulaire supérieure telle que \( B = UB^* \) .
    est dite \textbf{propre} \footnote{Une base propre est aussi connue sous le nom de base size-réduite dans la littérature.} si

    \begin{equation}
        \displaystyle\max_{1 \leq i < j \leq n} |\mu_{i,j}| \leq \frac{1}{2}.
    \end{equation}
\end{definition}

\begin{counterexample}
    Bien que la proprification impose une certaine contrainte sur les coefficients de projection, elle ne garantit pas à elle seule que les vecteurs de la base soient presque orthogonaux.
    \begin{figure}[H]
        \centering
        \includegraphics[width=0.23\textwidth]{images/lattice_0_10.png}
        \caption{ Base propre mais peu orthogonale }
        \label{fig:nosufficient}
    \end{figure}
\end{counterexample}

Idéalement, nous souhaiterions trouver une base \( (\bb_i)_{1 \leq i \leq n} \) du réseau \( \LL \) telle que :

\[
    \| \bb_1 \| = \lambda_1(\LL), \quad
    \| \bb_2 \| = \lambda_2(\LL), \quad \ldots, \quad
    \| \bb_n \| = \lambda_n(\LL)
\]
Ceci implique \( \| \bb_1 \| \leq \cdots \leq \| \bb_n \|\), mais cela est trop difficile de trouver une telle base car cela reviendrait à résoudre SIVP.

\begin{definition}
    Une base \( (\bb_i)_{1 \leq i \leq m} \) satisfait la \textbf{condition de Lovász} \footnote{Une condition plus générale : \( ( \delta - \mu_{i+1,i}^2 ) \, \|\bb_i^*\|^2 \leq \|\bb_{i+1}^*\|^2 \text{ pour } 1 \leq i \leq n, \text{ où } \delta \in \left] \frac{1}{4}, 1 \right]  \) } si:
    \[
        \|\bb_i^*\|^2 \leq 2\|\bb_{i+1}^*\|^2 \quad \text{ pour tout } 1 \leq i < n
    \]
\end{definition}

\begin{remark}
    On peut interpréter cette condition comme une forme de quasi-croissance des normes \( \|\bb_i^*\| \) : elle n'exige pas que celles-ci soient strictement croissantes, mais impose que toute éventuelle décroissance soit contrôlée, autrement dit, qu'elles ne décroissent pas trop rapidement.
\end{remark}

Dès lors, il est naturel de se demander pourquoi ne pas échanger les vecteurs lorsque cette condition de Lovász n’est pas satisfaite.


\begin{example}
    Si l’on applique cette idée à l’exemple précédent, alors après permutation des vecteurs concernés et une nouvelle phase de réduction, on obtient à nouveau une base améliorée :
    \begin{figure}[H]
        \centering
        \includegraphics[width=0.23\textwidth]{lattice_1_10.png}
        \caption{ Nouvelle base size-réduite}
        \label{fig:sufficient}
    \end{figure}
\end{example}


\begin{definition}
    Une base \( \mathcal{B} \) d'un réseau \( \LL (\mathcal{B}) \)  est dite \textbf{LLL-réduite} si
    \begin{itemize}
        \item[$\bullet$] $\mathcal{B}$ est propre.
        \item[$\bullet$] $\mathcal{B}$ satisfait la condition de Lovàsz.
    \end{itemize}
\end{definition}

\begin{remark}
    Chaque vecteur de la base réduite a une norme au moins égale à la moitié de celle du précédent, garantissant ainsi une décroissance modérée.
\end{remark}

\begin{theoreme}
    Soit \( \mathcal{B} \) une base réduite du réseau \( \LL \subseteq \R^n \) et soit \( v \in \LL \setminus \{0\} \). Alors
    \[ \|\bb_1\| \leq 2^{(n-1)/2} \cdot \|\bv\| \]
\end{theoreme}

En particulier, ce résultat s’applique à un vecteur \( \bv \in \LL \) de plus petite norme non nulle, c’est-à-dire un vecteur atteignant \( \lambda_1(\LL) \). On en déduit donc :
\[
    \|\bb_1\| \leq 2^{(n-1)/2} \cdot \lambda_1(\LL),
\]
ce qui montre que \( \mathrm{LLL} \) fournit en temps polynomial un vecteur de norme à un facteur \( 2^{(n-1)/2} \) près du plus court vecteur du réseau.
Autrement dit, l’algorithme \( \mathrm{LLL} \) résout approximativement le problème du plus court vecteur (\( \mathrm{SVP} \)) avec un facteur d’approximation \( \gamma = 2^{(n-1)/2} \).
Par extension, en renvoyant les vecteurs de la base réduite, \( \mathrm{LLL} \) permet également de résoudre le problème \( \mathrm{SIVP} \) (\textit{Shortest Independent Vectors Problem}) avec le même facteur d’approximation.