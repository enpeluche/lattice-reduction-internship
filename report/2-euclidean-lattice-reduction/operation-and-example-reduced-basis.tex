\section{Fonctionnement et exemple}

Nous présentons à présent l’algorithme de Lenstra–Lenstra–Lovász (\( \mathrm{LLL} \)), dans sa forme classique. Originellement \( \mathrm{LLL} \) est apparu dans l'article de 1982 et servait à factoriser des polynômes à coefficients rationnels.

\begin{smallalgo}{LLL}{algo:LLL_MCA}
    \LinesNumbered
    \DontPrintSemicolon
    \KwIn{Une base \( B = (\bb_1, \ldots, \bb_n) \)}
    \KwOut{Une base réduite \( G = (\bg_1, \ldots, \bg_n) \) de \( B \)}

    \For{\( i = 1 \) \KwTo \( n \)}{
        \( \bg_i \leftarrow \bb_i \)\;
    }

    \( (B^*, U) \leftarrow \) \textsc{Gram-Schmidt} \( (B) \)\;

    \While{\( i \leq n \)}
    {
        \For{\( j = i-1, i-2, \ldots, 1 \)}{
            \( \bg_i \leftarrow \bg_i - \nint{\mu_{i,j}} \, \bg_j \)\;
            Mettre à jour \( B^*, U  \)\;
        }

        \If{$i > 1$ \textbf{et} $\|\bg_{i-1}^*\|^2 > 2 \|\bg_{i}^*\|^2$}{
            Échanger \( \bg_{i-1} \) et \( \bg_i \)\;
            Mettre à jour \( B^*, U  \)\;
            \( i \leftarrow i - 1 \)\;
        }
        \Else{
            \( i \leftarrow i + 1 \)\;
        }
    }

    \KwRet{\( G = (\bg_1, \ldots, \bg_n) \)}
\end{smallalgo}

\vspace{1cm}
L’algorithme débute par le calcul de la base orthogonalisée de Gram–Schmidt (\textbf{ligne 3}), qui sert de support aux opérations de réduction.
Le principe général repose sur l’application répétée de deux types d’étapes : des \emph{réductions de taille} (\textbf{lignes 5 à 7}) visant à raccourcir les vecteurs sans sortir du réseau, et des \emph{permutations} de vecteurs (\textbf{ligne 8}) effectuées lorsque la condition de Lovász, qui contrôle la décroissance des normes, n’est pas satisfaite.

L’ensemble de ces opérations est imbriqué dans une boucle \texttt{while} qui se répète tant qu’une condition de progression n’est pas remplie.
Nous verrons que cette condition de Lovász garantit non seulement une amélioration à chaque étape, mais également la terminaison de l’algorithme en un nombre fini d’itérations.

\begin{example}
    Soit
    \[
        B =
        \begin{pmatrix}
            1  & 1 & 1 \\
            -1 & 0 & 2 \\
            3  & 5 & 6
        \end{pmatrix}
        \in GL_3(\Z).
    \]
    On a $\displaystyle |\LL (B)| = 9, \quad A = \max_{1 \leq i \leq 3} \| \bb_i \| = 70$

    On va essayer d'estimer un plus court vecteur
    L'algorithme \hyperref[algo:LLL_MCA]{\emph{LLL}} commence par calculer la décomposition de Gram-Schmidt de \( B \), pour plus de détails sur le calcul de cette décomposition, ce calcul est effectué dans l'annexe Rappels d'algèbres linéaire.

    On obtient la décomposition

    \[
        \begin{aligned}
            U & =
            \begin{pmatrix}
                1            & 0             & 0 \\
                \frac{1}{3}  & 1             & 0 \\
                \frac{14}{3} & \frac{13}{14} & 1
            \end{pmatrix}, \quad
            B^* =
            \begin{pmatrix}
                1            & 1             & 1              \\
                -\frac{4}{3} & - \frac{1}{3} & \frac{5}{3}    \\
                -\frac{3}{7} & \frac{9}{14}  & - \frac{3}{14}
            \end{pmatrix}.
        \end{aligned}
    \]

    Voici un tableau récapitulant les principales étapes de l'algorithme. Le tableau est volontairement détaillé et fourni, le lecteur pourra revenir sur cet exemple pour comprendre ce que sont $d_1$, $d_2$, $D$ ou la signification de $\Gram(G)$.

    \setlength{\tabcolsep}{0pt}

    \begin{center}
        \begin{tabular}{|c|c|c|c|c|c|} \hline % Premiere ligne
            \( G \)
             &
            \( U \)
             &
            \( G^* \)
             &
            \parbox[c][1cm][c]{1cm}
            {
                \centering
            \shortstack{\( d_1, d_2 \) \\ \( D \)}
            }
             &
            \(
            \begin{pmatrix}
                \|\bg_1^*\|^2 \\
                \|\bg_2^*\|^2 \\
                \|\bg_3^*\|^2
            \end{pmatrix}
            \)
             &
            \( \Gram(G) \)
            \\
            \hline \tikzmark{l1} % Deuxieme ligne
            \parbox[c][2.2cm][c]{3cm}
            {%
                \[
                    \begin{pmatrix}
                        1\quad 1\quad 1  \\
                        -1\quad 0\quad 2 \\
                        3\quad 5\quad 6
                    \end{pmatrix}
                \]
            }
             &
            \(
            \begin{pmatrix}
                1            & 0             & 0 \\
                \frac{1}{3}  & 1             & 0 \\
                \frac{14}{3} & \frac{13}{14} & 1
            \end{pmatrix}
            \)
             &
            \(
            \begin{pmatrix}
                1            & 1            & 1              \\
                -\frac{4}{3} & -\frac{1}{3} & \frac{5}{3}    \\
                -\frac{3}{7} & \frac{9}{14} & - \frac{3}{14}
            \end{pmatrix}
            \)
             &
            \parbox[c][0.9cm][c]{0.9cm}
            {
                \centering
            \shortstack{\( 3, 14 \)    \\ \( 42 \)}
            }
             &
            \(
            \begin{pmatrix}
                3            \\
                \frac{14}{3} \\
                \frac{9}{14}
            \end{pmatrix}
            \)
             &
            \(
            \begin{pmatrix}
                3  & 1 & 14 \\
                1  & 1 & 9  \\
                14 & 9 & 70
            \end{pmatrix}
            \)
            \\
            \hline \tikzmark{l2} % troisieme ligne
            \parbox[c][2.2cm][c]{3cm}
            {%
                \[
                    \begin{pmatrix}
                        1\quad 1\quad 1  \\
                        -1\quad 0\quad 2 \\
                        0\quad 1\quad 0
                    \end{pmatrix}
                \]
            }
             &
            \(
            \begin{pmatrix}
                1           & 0             & 0 \\
                \frac{1}{3} & 1             & 0 \\
                \frac{1}{3} & \frac{-1}{14} & 1
            \end{pmatrix}
            \)
             &
            \(
            \begin{pmatrix}
                1            & 1            & 1              \\
                -\frac{4}{3} & -\frac{1}{3} & \frac{5}{3}    \\
                -\frac{3}{7} & \frac{9}{14} & - \frac{3}{14}
            \end{pmatrix}
            \)
             &
            \parbox[c][0.9cm][c]{0.9cm}
            {
                \centering
            \shortstack{\( 3, 14 \)    \\ \( 42 \)}
            }
             &
            \(
            \begin{pmatrix}
                3            \\
                \frac{14}{3} \\
                \frac{9}{14}
            \end{pmatrix}
            \)
             &
            \(
            \begin{pmatrix}
                3 & 1 & 1 \\
                1 & 5 & 0 \\
                1 & 0 & 1
            \end{pmatrix}
            \)
            \\ \hline \tikzmark{l3}

            \parbox[c][1.6cm][c]{2.5cm}
            {%
                \centering
                \[
                    \begin{pmatrix}
                        1  & 1 & 1 \\
                        0  & 1 & 0 \\
                        -1 & 0 & 2
                    \end{pmatrix}
                \]
            }
             &
            \(
            \begin{pmatrix}
                1           & 0            & 0 \\
                \frac{1}{3} & 1            & 0 \\
                \frac{1}{3} & \frac{-1}{2} & 1
            \end{pmatrix}
            \)
             &
            \parbox[c][2.2cm][c]{3cm}
            {%

                \(
                \begin{pmatrix}
                    1            & 1           & 1            \\
                    -\frac{1}{3} & \frac{2}{3} & -\frac{1}{3} \\
                    -\frac{3}{2} & 0           & \frac{3}{2}
                \end{pmatrix}
                \)
            }
             &
            \parbox[c][0.9cm][c]{0.9cm}
            {
                \centering
            \shortstack{\( 3, 2 \)     \\ \( 6 \)}
            }
             &
            \(
            \begin{pmatrix}
                3           \\
                \frac{2}{3} \\
                \frac{9}{2}
            \end{pmatrix}
            \)
             &
            \(
            \begin{pmatrix}
                1           & 0            & 0 \\
                0           & 1            & 0 \\
                \frac{1}{3} & \frac{-1}{2} & 1
            \end{pmatrix}
            \)
            \\
            \hline  \tikzmark{l4}
            \parbox[c][2.2cm][c]{3cm}
            {%
                \[
                    \begin{pmatrix}
                        0  & 1 & 0 \\
                        1  & 1 & 1 \\
                        -1 & 0 & 2
                    \end{pmatrix}
                \]
            }
             &
            \(
            \begin{pmatrix}
                1           & 0            & 0 \\
                1           & 1            & 0 \\
                \frac{1}{3} & \frac{-1}{2} & 1
            \end{pmatrix}
            \)
             &
            \(
            \begin{pmatrix}
                0            & 1 & 0           \\
                1            & 0 & 1           \\
                -\frac{3}{2} & 0 & \frac{3}{2}
            \end{pmatrix}
            \)
             &
            \parbox[c][0.9cm][c]{0.9cm}
            {
                \centering
            \shortstack{\( 1, 2 \)     \\ \( 2 \)}
            }
             &
            \(
            \begin{pmatrix}
                1 \\
                2 \\
                \frac{9}{2}
            \end{pmatrix}
            \)
             &
            \(
            \begin{pmatrix}
                1           & 0            & 0 \\
                0           & 1            & 0 \\
                \frac{1}{3} & \frac{-1}{2} & 1
            \end{pmatrix}
            \)
            \\
            \hline \tikzmark{l5}
            \parbox[c][2.2cm][c]{3cm}
            {%     
                \[
                    \begin{pmatrix}
                        0  & 1 & 0 \\
                        1  & 0 & 1 \\
                        -1 & 0 & 2
                    \end{pmatrix}
                \]
            }
             &
            \(
            \begin{pmatrix}
                1           & 0            & 0 \\
                0           & 1            & 0 \\
                \frac{1}{3} & \frac{-1}{2} & 1
            \end{pmatrix}
            \)
             &
            \(
            \begin{pmatrix}
                0            & 1 & 0           \\
                1            & 0 & 1           \\
                -\frac{3}{2} & 0 & \frac{3}{2}
            \end{pmatrix}
            \)
             &
            \parbox[c][0.9cm][c]{0.9cm}
            {
                \centering
            \shortstack{\( 1, 2 \)     \\ \( 2 \)}
            }
             &
            \(
            \begin{pmatrix}
                1 \\
                2 \\
                \frac{9}{2}
            \end{pmatrix}
            \)
             &
            \(
            \begin{pmatrix}
                1 & 0 & 0 \\
                0 & 2 & 1 \\
                0 & 1 & 5
            \end{pmatrix}
            \)
            \\ \hline
        \end{tabular}
    \end{center}

    \vspace{1cm}

    On obtient la base  LLL réduite :

    \[
        \bg_{reduced}=
        \begin{pmatrix}
            0  & 1 & 0 \\
            1  & 0 & 1 \\
            -1 & 0 & 2
        \end{pmatrix}
        \in M_3(\Z)
    \]



\end{example}


\begin{tikzpicture}[remember picture, overlay]
    \draw[->, thick, bend right=80]
    ([xshift=0em]pic cs:l1) to
    node[midway, left]{\small \shortstack{proprification\\ $\bg_3$}}
    ([xshift=0em, yshift=1em]pic cs:l2);

    \draw[->, thick, bend right=80]
    ([xshift=0em]pic cs:l2) to
    node[midway, left]{\small \shortstack{Lovasz\\ $\bg_2 \leftrightarrow \bg_3$}}
    ([xshift=0em, yshift=1em]pic cs:l3);

    \draw[->, thick, bend right=80]
    ([xshift=0em]pic cs:l3) to
    node[midway, left]{\small \shortstack{Lovasz\\ $\bg_1 \leftrightarrow \bg_2$}}
    ([xshift=0em, yshift=1em]pic cs:l4);

    \draw[->, thick, bend right=80]
    ([xshift=0em]pic cs:l4) to
    node[midway, left]{\small \shortstack{proprification\\ $\bg_2$}}
    ([xshift=0em, yshift=1em]pic cs:l5);
\end{tikzpicture}

\begin{remark}
    \leavevmode\vspace{0.5\baselineskip}
    \begin{itemize}
        \item[$\bullet$] La valeur $d_3$ ne nous intéresse pas car il s'agit d'un invariant,
        \item[$\bullet$] \(Gram(G)\) se rapproche petit a petit d'une matrice diagonale.
    \end{itemize}
\end{remark}
