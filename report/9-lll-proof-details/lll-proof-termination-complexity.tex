
\section{Terminaison et complexité}
\begin{theoreme}[Terminaison et complexité]
    On pose \( \displaystyle A = \max_{1 \leq i \leq n} \| \bg_i \| \). L'algorithme \hyperref[algo:LLL_MCA]{\emph{LLL }} termine et utilise \( \OO(n^4 \log A) \) opérations arithmétiques sur des entiers.
\end{theoreme}

La difficulté est de montrer que la boucle Tant que ne va pas s'exécuter indéfiniment.

\begin{lemma}
    \leavevmode\vspace{0.3\baselineskip}
    \begin{enumerate}
        \item Orthogonalisation de Gram-Schmidt nécessite \( \OO(n^3) \) opérations dans \( \Z \).

        \item \hyperref[step:P]{\emph{Proprification de \( \bg_i \)}} nécessite \( \OO(n^2) \) opérations dans \( \Z \)

        \item \hyperref[step:R]{\emph{Réduction de \( \bg_{i-1}, \bg_{i} \)}} nécessite \( \OO(n) \) opérations dans \( \Z \)
    \end{enumerate}
\end{lemma}

\begin{proof}
    \begin{enumerate}
        \item Il faut utiliser le théorème sur la complexité de l'algorithme de Gram-Schmidt.

        \item Une exécution de \hyperref[step:P]{\emph{Proprification partielle de \( \bg_i \)}} revient à faire les multiplications \( H=EG \) et \( N=EM \) se font en \( \OO(n) \) étapes, ainsi une exécution de \hyperref[step:P]{\emph{Proprification de \( \bg_i \)}} nécessite \( \OO(n^2) \) opérations dans \( \Z \).

        \item Si un échange a lieu à \hyperref[step:R]{\emph{Réduction de $\bg_{i-1}, \bg_{i}$}}, alors seuls \( g^*_{i-1} \), \( g^*_i \), ainsi que les lignes et colonnes \( i-1 \) et \( i \) de la matrice de transition \( M \) sont modifiés, et ces éléments peuvent être mis à jour en \( O(n) \) opérations.
    \end{enumerate}
\end{proof}

Il reste à borner le nombre d'itérations de la boucle Tant que à l'étape \hyperref[step:LLL]{\emph{LLL}}.

Pour tout \( 1 \leq k \leq n \), on pose
\[
    \bg_k =
    \begin{pmatrix}
        \bg_1  \\
        \vdots \\
        \bg_k
    \end{pmatrix}
    \in \Z^{k \x n}
    , \quad d_0=1, \quad d_k = \det(\bg_k \cdot \bg_k^T) \in \Z.
\]

\begin{lemma}
    Pour tout \( 1 \leq k \leq n \), on a :
    \[
        d_k = \prod_{1 \leq l \leq k} \| \bg_l^* \|^2 > 0.
    \]
\end{lemma}

\begin{proof}
    Soit \( 1 \leq k \leq n \), on définit \( \bg_k =  U_k \bg_k^* \) la décomposition de Gram-Schmidt de la famille \( (\bg_i) _{1 \leq i \leq k} \)

    Alors
    \[
        d_k = \det(\bg_k\bg_k^t)=\det(U_k \bg_k^* (\bg_k^*)^t U_k^t) = \det(\bg_k^* (\bg_k^*)^t) = \prod_{1 \leq l \leq k} \| \bg_l^* \|^2 > 0
    \]
\end{proof}
\begin{lemma}
    \leavevmode\vspace{0.5\baselineskip}
    \begin{enumerate}
        \item \hyperref[step:P]{\emph{Proprification de \( \bg_i \)}} ne change pas \( d_k \)  pour tout \( 1 \leq k \leq n \).

        \item Si \( \bg_{i-1} \) et \( \bg_i \) sont échangés à l’étape \hyperref[step:P]{\emph{Réduction de \( \bg_{i-1}, \bg_{i} \)}}, et si \( d_k^* \) désigne la nouvelle valeur de \( d_k \), alors :
              \[
                  d_k^* = d_k \quad \text{pour tout } k \neq i - 1, \quad \text{et} \quad d_{i-1}^* \leq \frac{3}{4} d_{i-1}.
              \]
    \end{enumerate}
\end{lemma}

\begin{proof}
    \begin{enumerate}
        \item D'après le lemme 2.2 \hyperref[step:P]{\emph{Proprification de \( \bg_i \)}} ne modifie pas \( \bg_k^* \) et donc ne modifie pas \( d_k \).

        \item Pour \( k \neq i-1\), une exécution de \hyperref[step:P]{\emph{Réduction de \( \bg_{i-1}, \bg_{i} \)}} multiplie \( \bg_k \) par une matrice de permutation, donc \( d_k^* = d_k \)

              De plus, on a
              \[
                  d_{i-1} \eqjust{2.6} \prod_{1 \leq l \leq i-1} \| \bg_l^* \|^2 \leqjust{2.3} \frac{3}{4} \prod_{1 \leq l \leq i-1} \| h_l^* \|^2 \eqjust{2.6} \frac{3}{4} d_{i-1}^*
              \]
    \end{enumerate}
\end{proof}
On pose
\[
    D = \prod_{1 \leq k < n} d_k, \quad \displaystyle A = \max_{1 \leq i \leq n} \| f_i \|
\]

On désigne \( D_0 \) désigne la valeur de \( D \) au début de l'algorithme, on a \( 1 \leq D \in \Z \) et

\[
    \begin{aligned}
        D_0 & = \|g^*_1\|^{2(n-1)} \|g^*_2\|^{2(n-2)} \cdots \|g^*_{n-1}\|^2    \\
            & \leq \|\bg_1\|^{2(n-1)} \|\bg_2\|^{2(n-2)} \cdots \|\bg_{n-1}\|^2 \\
            & \leq A^{n(n-1)}
    \end{aligned}
\]


Puisque \( f^*_i \) est une projection de \( f_i \) pour tout \( i \).

\begin{lemma}
    \leavevmode\vspace{0.5\baselineskip}
    \begin{enumerate}
        \item \hyperref[step:P]{\emph{Proprification de \( \bg_i \)}} ne modifie pas \( D \).
        \item \( D \) diminue d’au moins un facteur \( 3/4 \) si un échange a lieu dans \hyperref[step:R]{\emph{Réduction de \( \bg_{i-1}, \bg_{i} \)}}
    \end{enumerate}
\end{lemma}

\begin{proof}
    \begin{enumerate}
        \item D'après le lemme 2.7 \hyperref[step:P]{\emph{Proprification de \( \bg_i \)}} ne modifie pas \( d_k \) et donc ne modifie pas \( D \).

        \item Si \( \bg_{i-1} \) et \( \bg_i \) sont échangés lors de l'exécution de \hyperref[step:R]{\emph{Réduction de \( \bg_{i-1}, \bg_{i} \)}}, en notant \( D^* \) la nouvelle valeur de \( D \), alors d'après le lemme 2.7

              \[
                  d_k^* = d_k, \quad d_{i-1}^* \leq \frac{3}{4} d_{i-1} \text{ donc } D^* \leq \frac{3}{4} D.
              \]
    \end{enumerate}
\end{proof}

À tout moment de l’algorithme, soit \( e \in \N \) le nombre d’échanges effectués jusqu’à présent, et \( e^* \) le nombre de fois où la branche alternative (le \textit{else}) dans \hyperref[step:R]{\emph{Réduction de \( \bg_{i-1}, \bg_{i} \)}} a été prise.

\begin{lemma}
    On a
    \[
        e \leq \log_{4/3} D_0 \in \OO(n^2 \log A)
    \]
\end{lemma}

\begin{proof}
    Soit \( D_e \) la valeur de \( D \) après \( e \) échanges.

    On doit avoir
    \[
        1 \;\le\; D_e \; \le\;\left(\frac34\right)^{e} D_0 \le\; \left(\frac34\right)^{e}A^{\,n(n-1)}.
    \]

    En appliquant \( lo\bg_{3/4} \), aux extrémités de l'inégalité.

    \[
        0=\log_{3/4}(1) \;\ge\; e + \log_{3/4}(A^{\,n(n-1)})=e + n(n-1)\frac{\log A}{\log(3/4)}.
    \]

    On en déduit que \( e \; \leq n(n-1)\frac{\log A}{-\log(3/4)}\) et donc \( e \in \OO(n^2 \log A)\)

\end{proof}

\begin{proof}[Preuve de la terminaison et la complexité]

    Comme \( i \) est décrémenté de \( 1 \) lors d’un échange et incrémenté de \( 1 \) sinon l'entier \( i + e - e^* \) est constant tout au long de \hyperref[step:LLL]{\emph{LLL}}.

    Initialement \( i + e - e^* = 2 \) et à la fin de \hyperref[step:LLL]{\emph{LLL}} on a \( n + 1 + e - e^* = 2 \).
    On en déduit donc que \( e + e^* = 2e + n - 1 \in \OO(n^2 \log A) \).
    et donc d'après le lemme 2.5 le coût total de \hyperref[step:LLL]{\emph{LLL}} est \( \OO(n^2 \x n^2 \log A) \) opérations dans \( \Z \). Ce qui acheve la preuve.
\end{proof}

\begin{theoreme}
    L'algorithme \hyperref[algo:LLL_MCA]{\emph{LLL (LLL)}} opère sur des entiers dont la longueur est \( \OO(n \log A) \).
\end{theoreme}
Il reste à montrer la dernière partie du théorème.

\begin{lemma}
    Soit \( \bg_1, \dots, \bg_n \in \Z^n \), et soit \( G^* \) et \( M \) respectivement la base de Gram-Schmidt et la matrice des coefficients associés. Pour tout \( 1 \leq l < k \leq n \), on a :
    \begin{enumerate}[label=(\roman*)]
        \item \( d_{k-1} \bg_k^* \in \Z^n \)
        \item \( d_l \mu_{k,l} \in \Z \)
        \item \( |\mu_{k,l}| \leq \sqrt{d_{l-1}} \| \bg_k \| \)
    \end{enumerate}
\end{lemma}
\begin{proof}

    \begin{enumerate}
        \item On écrit
              \[
                  \bg_k^* = \bg_k - \sum_{1 \leq l < k}\lambda_{k,l} \bg_l, \quad \lambda \in \R.
              \]

              Soit \( j < k\). On a
              \[
                  0 = \ps{\bg_k^*}{\bg_j} = \ps{ \bg_k - \sum_{1 \leq l < k} \lambda_{k,l} \bg_l}{\bg_j}.
              \]
              Ce qui implique
              \[
                  \ps{\bg_k}{\bg_j} = \sum_{1 \leq l < k} \lambda_{k,l} \ps{\bg_l}{\bg_j}
              \]

              On a donc
              \[
                  \begin{pmatrix}
                      \ps{\bg_1}{\bg_1}       & \cdots & \ps{\bg_{k-1}}{\bg_1}     \\
                      \vdots                  &        & \vdots                    \\
                      \ps{\bg_{1}}{\bg_{k-1}} & \cdots & \ps{\bg_{k-1}}{\bg_{k-1}}
                  \end{pmatrix}
                  \begin{pmatrix}
                      \lambda_{k,1} \\
                      \vdots        \\
                      \lambda_{k,k-1}
                  \end{pmatrix}
                  =
                  \begin{pmatrix}
                      \ps{\bg_{k}}{\bg_{1}} \\
                      \vdots                \\
                      \ps{\bg_{k}}{\bg_{k-1}}
                  \end{pmatrix}
              \]

              D'après la règle de Cramer (à citer) on a

              \[
                  d_{k-1} \lambda_{k,1} =
                  \frac{
                      \begin{vmatrix}
                          \ps{\bg_1}{\bg_1}       & \cdots & \ps{\bg_{k-1}}{\bg_1}     \\
                          \vdots                  &        & \vdots                    \\
                          \ps{\bg_{1}}{\bg_{k-1}} & \cdots & \ps{\bg_{k-1}}{\bg_{k-1}}
                      \end{vmatrix}
                  }{
                      \begin{vmatrix}
                          \ps{\bg_1}{\bg_1}       & \cdots & \ps{\bg_{k-1}}{\bg_1}     \\
                          \vdots                  &        & \vdots                    \\
                          \ps{\bg_{1}}{\bg_{k-1}} & \cdots & \ps{\bg_{k-1}}{\bg_{k-1}}
                      \end{vmatrix}} det(\bg_k \bg_k^t)
                  = \begin{vmatrix}
                      \ps{\bg_1}{\bg_1}       & \cdots & \ps{\bg_{k-1}}{\bg_1}     \\
                      \vdots                  &        & \vdots                    \\
                      \ps{\bg_{1}}{\bg_{k-1}} & \cdots & \ps{\bg_{k-1}}{\bg_{k-1}}
                  \end{vmatrix} \in \Z
              \]

        \item
              \[
                  d_l \mu_{k,l} = d_l \frac{\ps{\bg_k}{\bg_l^*}}{\|\bg_l^*\|^2} = d_l \frac{\ps{\bg_k}{\bg_l^*}}{d_l/d_{l-1}} = d_{l-1} \ps{\bg_k}{\bg_l^*}=\ps{\bg_k}{\bg_l^* d_{l-1}} \in \Z
              \]


        \item
              \[
                  | \mu_{k,l} | = \frac{ \ps{\bg_k}{\bg_l^*} }{\|\bg_l^*\|}^2 \leq \frac{\|\bg_k\|}{\|\bg_l^*\|} \leq \sqrt{d_{l-1}/d_{l}} \|\bg_k\|\leq \sqrt{d_{l-1}} \|\bg_k\|
              \]
    \end{enumerate}

\end{proof}


Nous avons supposé que \( \|\bg_k\| \leq A \) pour tout \( k \). Alors \( A \) est également une borne supérieure pour la base orthogonale de Gram-Schmidt initiale : \( \|\bg_k^*\| \leq A \) pour tout \( k \).


On a d'après les lemmes \( \max \left\{ \|\bg_k^*\| : 1 \leq k \leq n \right\} \) ne croît jamais au cours de l'algorithme. Ainsi, à tout instant et pour tout \( k \), on a :
\[
    \|\bg_k^*\| \leq A \quad \text{et} \quad d_k = \prod_{1 \leq l \leq k} \|\bg_l^*\|^2 \leq A^{2k}.
\]

\begin{lemma}
    Soit \( 1 \leq k \leq n \).

    \begin{enumerate}
        \item À tout moment de l’algorithme, sauf éventuellement à l'étape \hyperref[step:P]{\emph{Proprification de \( \bg_i \)}} lorsque \( k = i \), on a :
              \[
                  \|\bg_k\| \leq \sqrt{n} A.
              \]

        \item À chaque exécution de l’étape \hyperref[step:P]{\emph{Proprification partielle de \( \bg_i \)}}, on a :
              \[
                  \|\bg_i\| \leq n(2A)^n.
              \]
    \end{enumerate}
\end{lemma}

\begin{proof}
    \begin{enumerate}
        \item Initialement \( \|\bg_k\| \leq A \) pour tout \( k \). L'étape \hyperref[step:P]{\emph{Réduction de \( \bg_{i-1}, \bg_{i} \)}} ne modifie pas \( \|\bg_k\| \), il suffit donc d'examiner  l'étape \hyperref[step:P]{\emph{Proprification de \( \bg_i \)}}. On a que \( \bg_k \), pour \( k \neq i \), n'est pas affecté par l'étape  \hyperref[step:P]{\emph{Proprification partielle de \( \bg_i \)}}.

              Soit \( m_i = \max \{ |\mu_{i,l}| : 1 \leq l \leq i \} \). À partir de
              \[
                  \bg_i = \sum_{1 \leq l \leq i} \mu_{i,l} g^*_l
              \]
              et de l'orthogonalité des vecteurs \( g^*_l \), on obtient :
              \[
                  \|\bg_i\|^2 = \sum_{1 \leq l \leq i} \mu_{i,l}^2 \|g^*_l\|^2 \leq n m_i^2 A^2,
                  \quad \text{donc} \quad
                  \|\bg_i\| \leq \sqrt{n}\, m_i A. \tag{4}
              \]

              À la fin de \hyperref[step:P]{\emph{Proprification de \( \bg_i \)}}, on a \( m_i = 1 \) par le lemme 2.1.

        \item  Le lemme 2.10 et le point 1 impliquent qu'au début de \hyperref[step:P]{\emph{Proprification de \( \bg_i \)}}, on a
              \[
                  \begin{aligned}
                      m_i & \leq \max \left\{ d_l^{1/2} : 1 \leq l \leq i \right\} \cdot \|\bg_i\| \\
                          & \leq A^{n-2} \cdot n^{1/2} A                                           \\
                          & = n^{1/2} A^{n-1}.
                  \end{aligned}
              \]

              Considérons maintenant le remplacement effectué à l'étape  \hyperref[step:P]{\emph{Proprification partielle de \( \bg_i \)}}. Comme \( m_i \geq 1 \) et que \( |\mu_{j,l}| \leq \tfrac{1}{2} \) pour \( 1 \leq l < j \), le lemme 2.8 donne :
              \[
                  \begin{aligned}
                      |\mu_{i,l} - \lfloor \mu_{i,j} \rceil \mu_{j,l}|
                       & \leq |\mu_{i,l}| + |\lfloor \mu_{i,j} \rceil| \cdot |\mu_{j,l}| \\
                       & \leq m_i + \left(m_i + \tfrac{1}{2} \right) \cdot \tfrac{1}{2}  \\
                       & = \tfrac{3}{2} m_i + \tfrac{1}{4}                               \\
                       & \leq 2m_i
                  \end{aligned}
              \]
              pour \( 1 \leq l < j \).

              Pour \( l = j \), la nouvelle valeur de \( \mu_{i,j} \) est par construction au plus \( \tfrac{1}{2} \) en valeur absolue, tout comme les valeurs de \( \mu_{i,l} \) pour \( l > j \), d'après le lemme 2.1.

              On en déduit que pour chaque valeur de \( j \), la valeur de \( m_i \) est au plus doublée. Ainsi, pendant \hyperref[step:P]{\emph{Proprification de \( \bg_i \)}}, la valeur de \( m_i \) est multipliée au plus par un facteur \( 2^{i-1} \leq 2^{n-1} \).

              On a donc
              \[
                  m_i \leq n^{1/2}(2A)^{n-1}
              \]

              Puis
              \[
                  \|\bg_i\| \leq n^{1/2} m_i A \leq n (2A)^n.
              \]
    \end{enumerate}
\end{proof}



\begin{proof}[Preuve du theoreme]
    \begin{itemize}
        \item Les dénominateurs \( d_l \) des nombres rationnels calculés pendant l’algorithme sont au plus \( A^{2n} \), et leur taille est en \( \OO(n \log A) \).
        \item Les numérateurs sont majorés en valeur absolue par :
              \begin{itemize}
                  \item[$\bullet$] \( \|\bg_k\|_\infty \leq \|\bg_k\| \leq n(2A)^n \) d'après le lemme 2.11
                  \item[$\bullet$] \( \|d_{k-1} \bg_k^*\|_\infty \leq \|d_{k-1} \bg_k^*\| \leq A^{2k - 2} A \leq A^{2n} \) d'après le lemme 2.10
                  \item[$\bullet$] \( |d_l \mu_{k,l}| \leq d_l d_{l-1}^{1/2} \|\bg_l\| \leq A^{2l} A^{l-1} n (2A)^n \leq n (2A^4)^n \) d'après les lemmes 2.10 et 2.11
              \end{itemize}

              et par conséquent, leur taille est aussi en \( \OO(n \log A) \).
    \end{itemize}
\end{proof}