\usepackage[a4paper, left=2cm, right=2cm, top=2cm, bottom=2cm]{geometry}         % Définition des marges du document (A4 avec 2.5 cm sur chaque côté) %
%                                                                                                                                                             %
\usepackage[french]{babel}                                % Déclare deux langues : anglais et français, la dernière énumérée étant la langue principale %
%							                                    % Cela gère la césure, les titres automatiques de chapitres, etc.                             %
%												      		                                                                                                  %
\usepackage[utf8]{inputenc}                                     % Permet d'écrire directement des accents et caractères spéciaux avec l'encodage UTF-8        %
%                                                                                                                                                             %
\usepackage[T1]{fontenc}                                     % Définit l'encodage des polices pour une meilleure gestion des caractères                       %
%						                                     % T1 : Encodage qui améliore l'affichage des accents (é, à, ç...) et la césure correcte des mots %
%                                                            % Recommandé pour les documents en français                                                      %
\usepackage{lettrine}                                                       % Gestion de lettrine décorative au début d'un paragraphe                         %
%                                                                                                                                                             %
\usepackage{tgtermes} % Belle police compatible Times, un peu plus compacte
% Police times pour le texte (pas forcément pour les maths)                       %
%                                                                                                                                                             %
\usepackage{pifont}                                                         % Symboles dingbats (\ding{...})                                                  %
\newcommand{\xmark}{\ding{55}}                                              % Définit la macro \xmark qui affiche ✗                                           %
\usepackage{calc}                                                           % Fournit des capacités de calcul sur les longueurs et dimensions en LaTeX        %
\usepackage{csquotes}                                                       % Charge le package csquotes pour une gestion avancée des guillemets et citations %
\usepackage{xcolor}                                                         % La gestion avancée des couleurs                                                 %
\usepackage{titlesec}                                                                % Personnalisation des titres de sections, chapitres (eccecements, etc.) %
\titlespacing*{\chapter}{0pt}{0pt}{20pt}                                             % Réduction de l'espacement avant et après le chapitre                   %
\usepackage{fancyhdr}                                                                % En-têtes et pieds de page personnalisés                                %
\pagestyle{fancy}                                                                    % modification flexible de l'en-tête et du pied de page                  %
\fancyhf{}                                                                           % vide l'en-tête et le pied de page                                      %
\fancyhead[L]{\leftmark}                                                             % Section ou chapitre courant                                            %
\fancyfoot[C]{\thepage}                                                              % Numéro de page au centre                                               %
\setlength{\headheight}{30pt}
\usepackage{enumitem}                                  % Personnalisation des listes (itemize, enumerate), contrôle fin de l'indentation, etc.                %
\setlist[itemize]{topsep=2pt, partopsep=0pt, parsep=0pt, itemsep=2pt}
\setlist[enumerate]{topsep=2pt, partopsep=0pt, parsep=0pt, itemsep=2pt}
\usepackage{ragged2e}                                  % Permet des environnements de justification (alignement du texte, etc.)                               %
\usepackage{parskip}                                   % parskip et les commandes suivantes ajustent l'indentation et l'espacement vertical entre paragraphes %
\setlength{\parindent}{0pt}                            % aucune indentation                                                                                   %
\setlength{\parskip}{2pt}                              % ajoute un léger espacement vertical entre paragraphes                                                %
\titleformat{\subsubsection}[runin]          % Personnalisation de l'affichage des sous-sous-sections [runin] = titre sur la même ligne que le texte qui suit %
{\normalfont\large\bfseries}                 % style du titre                                                                                                 %
{\thesubsubsection}                          % numéro                                                                                                         %
{1em}                                                                                                                                                         %
{}                                                                                                                                                            %
\setcounter{secnumdepth}{3}                  % On numérote jusqu'aux subsubsections                                                                           %
\setcounter{tocdepth}{3}                     % On les inclus dans la table des matières                                                                       %
\titleformat
{\section}           % command
[hang]               % shape
{\large}             % format
{\Large\textbf{\thesection}  % label (plus de makebox ni rule)
}
{1ex}                % sep
{\Large\itshape}      % body format
[]
\titleformat
{\subsection}           % command
[hang]               % shape
{\normalsize}             % format
{\large\textbf{\thesubsection}  % label (plus de makebox ni rule)
}
{1ex}                % sep
{\large\itshape}      % body format
[]

\usepackage{subcaption}                                          % Gère des sous-légendes de figures                                                          %

\usepackage{tikz}
\usetikzlibrary{tikzmark,arrows.meta}
\usepackage{array}
\usepackage{makecell}
\usetikzlibrary{matrix,fit}
%                                                                                                                                                             %
\usepackage{graphicx}                                                       % Inclusion d'images avec \includegraphics                                        %
%                                                                                                                                                             %
\usepackage{float}                                                          % Placement plus précis des figures (avec l'option [H], etc.)                     %
%                                                                                                                                                             %
\usepackage{tikz}                                                           % tikz est un package de dessin vectoriel                                         %
%                                                                                                                                                             %
\usepackage{wrapfig}                                                        % wrapfig permet d'enrouler le texte autour d'images                              %
%                                                                                                                                                             %
\usepackage{eso-pic}                                                        % eso-pic permet de placer des éléments en arrière-plan (watermarks, logos, etc.) %
\usepackage{amsmath, amsthm, amssymb}                                                % amsmath :  Pour des équations et des notations mathématiques avancées. %
%									                                                 % amsthm : Pour formater des théorèmes, définitions, propositions, etc.  %
%									                                                 % amssymb : Pour ajouter des symboles mathématiques supplémentaires      %
\usepackage{mathtools}                                                               % Ajoute des fonctionnalités avancées à amsmath (p. ex. \mathclap, etc.) %
\usepackage{mathrsfs}                                                                % Fournit la police calligraphique \mathscr                              %
\usepackage{bm}                                                                      % Permet de mettre des caractères maths en gras (ex. \bm{\alpha})        %

\newcommand{\Z}{\mathbb{Z}}
\newcommand{\N}{\mathbb{N}}
\newcommand{\Q}{\mathbb{Q}}
\newcommand{\C}{\mathbb{C}}
\newcommand{\F}{\mathbb{F}}
\newcommand{\R}{\mathbb{R}}
\newcommand{\K}{\mathbb{K}}
\newcommand{\LL}{\mathscr{L}}
\newcommand{\bb}{\mathbf{b}}
\newcommand{\bg}{\mathbf{g}}
\newcommand{\by}{\mathbf{y}}
\newcommand{\bu}{\mathbf{u}}
\newcommand{\bv}{\mathbf{v}}
\newcommand{\bx}{\mathbf{x}}

\usepackage{tikz-cd}

\renewcommand{\bf}{\mathbf{f}}
\newcommand{\OO}{\mathcal{O}}

\DeclareMathOperator{\rdeg}{rdeg}
\DeclareMathOperator{\MM}{MM}
\DeclareMathOperator{\M}{M}
\DeclareMathOperator{\lcoeff}{lcoeff}
\DeclareMathOperator{\rang}{rang}
\DeclareMathOperator{\size}{size}
\DeclareMathOperator{\Gram}{Gram}
\providecommand{\x}{\times}

\newcommand{\eqjust}[1]{\overset{(#1)}{=}}
\newcommand{\leqjust}[1]{\overset{(#1)}{\leq}}
\newcommand{\nint}[1]{\left\lceil #1 \right\rfloor} % Nearest integer
\newcommand{\ps}[2]{\left\langle #1 , #2 \right\rangle}
\newcommand{\norm}[1]{\left\| #1 \right\|}
\newcommand{\gscoeff}[2]{\frac{\ps{#1}{#2}}{\norm{#2}^2} #2}

\usepackage[french,ruled,vlined]{algorithm2e}
\SetKwInput{KwIn}{\textbf{Entrée}}
\SetKwInput{KwOut}{\textbf{Sortie}}
\SetKw{KwRet}{\textbf{Retourner}}
\SetKw{KwIf}{\textbf{Si}}
\SetKw{KwElse}{\textbf{Sinon}}
\SetKw{KwElseIf}{\textbf{Sinon si}}
\SetKw{KwFor}{\textbf{Pour}}
\SetKw{KwWhile}{\textbf{Tant que}}
\SetKw{KwDo}{\textbf{faire}}
\SetKw{KwThen}{\textbf{alors}}
\SetKw{KwEnd}{\textbf{Fin}}
\SetKw{KwTo}{\textbf{à}}
\SetKwFor{While}{\textbf{Tant que}}{\textbf{faire}}{\textbf{Fin}}
\SetKwIF{If}{ElseIf}{Else}{\textbf{Si}}{\textbf{alors}}{\textbf{Sinon si}}{\textbf{Sinon}}{\textbf{Fin}}
\SetKwFor{For}{\textbf{Pour}}{\textbf{faire}}{\textbf{Fin}}
%
\usepackage{listings}                                                                  % Insertion de code                                                    %

\usepackage{float}       % pour [H]
\usepackage{algorithm2e} % pour l'environnement algorithm

\newenvironment{smallalgo}[2]{
	\begin{center}
		\begin{minipage}{0.75\linewidth}
			\begin{algorithm}[H]

				\caption{\textit{#1}}         % Titre en italique
				\label{#2}
				\vspace{4pt}
				\LinesNumbered
				\DontPrintSemicolon
				}{
			\end{algorithm}
		\end{minipage}
	\end{center}
}

\usepackage[backend=biber,style=authoryear]{biblatex}                                            % Gestion de la bibliographie par biblatex                   %
\setlength{\bibitemsep}{1\baselineskip}                                                          % Ajuste l'espacement entre les éléments de la bibliographie %
\addbibresource{references/references.bib}                                                                  % On inclue le fichier BibTeX references.bib                 %
\usepackage[most]{tcolorbox}
\tcbuselibrary{skins, breakable}
\usepackage{mdframed}

\newtheoremstyle{definitionstyle}                                                                                                % Nom du style               %
{10pt}                                                                                                                           % Espace avant               %
{10pt}                                                                                                                           % Espace après               %
{\normalfont}                                                                                                                    % Police normale             %
{}                                                                                                                               % Retrait                    %
{\bfseries}                                                                                                                      % Titre en gras              %
{.}                                                                                                                              % Ponctuation après le titre %
{1em}                                                                                                                            % Espace après le titre      %
{}                                                                                                                               % Personnalisation du titre  %
%                                                                                                                                                             %
\theoremstyle{definitionstyle}                                                                                                                                %
\newtheorem{definition}{Définition}[chapter]                                                                                     % Numérotation par chapitres %
%                                                                                                                                                             %
%                                                          EXEMPLE, CONTRE-EXEMPLE, REMARQUE                                                                  %
\newtheoremstyle{examplestyle}                                                                                                   % Nom du style               %
{10pt}                                                                                                                           % Espace avant               %
{10pt}                                                                                                                           % Espace après               %
{\normalfont}                                                                                                                    % Police normale             %
{}                                                                                                                               % Retrait                    %
{\itshape}                                                                                                                       % Titre en italique          %
{.}                                                                                                                              % Ponctuation après le titre %
{1em}                                                                                                                            % Espace après le titre      %
{}                                                                                                                               % Personnalisation du titre  %
%                                                                                                                                                             %
\theoremstyle{examplestyle}                                                                                                                                   %
\newtheorem*{example}{Exemple}                                                                                                   % Aucune numérotation        %
\newtheorem*{counterexample}{Contre exemple}                                                                                     % Aucune numérotation        % 
\newtheorem*{notation}{Notation}                                                                                                 % Aucune numérotation        % 
%                                                                                                                                                             %
\newtheorem*{remark}{Remarque}                                                                                                   % Aucune numérotation        %
\newtheoremstyle{propositionstyle}                                                                                             % Nom du style                 %
{10pt}                        	                                                                                               % Espace avant                 %
{10pt}                        	                                                                                               % Espace après                 %
{\normalfont}                 	                                                                                               % Police normale pour le texte %
{}                            	                                                                                               % Retrait                      %
{\bfseries}                    	                                                                                               % Titre en italique            %
{.}                           	                                                                                               % Ponctuation après le titre   %
{1em}                                                                                                                          % Espace après le titre        %
{}                             	                                                                                               % Pas d'autres spécifications  %
\usepackage[most]{tcolorbox}
\newcounter{problem}
\usepackage{xparse}
\usepackage{lipsum}

\def\exampletext{Problème} % Mot avant le titre, ici : "problème"

\NewDocumentEnvironment{problem}{ O{} }
{
	%red!55!black
	\colorlet{colexam}{black} % Global example color
	\newtcolorbox[use counter=problem]{problembox}{%
		% Example Frame Start
		empty,% Empty previously set parameters
		title={#1},% use \thetcbcounter to access the testexample counter text
		% Attaching a box requires an overlay
		attach boxed title to top left,
		% Ensures proper line breaking in longer titles
		minipage boxed title,
		% (boxed title style requires an overlay)
		boxed title style={empty,size=minimal,toprule=0pt,top=4pt,left=3mm,overlay={}},
		coltitle=colexam,fonttitle=\itshape,
		before=\par\medskip\noindent,parbox=false,boxsep=0pt,left=3mm,right=0mm,top=2pt,breakable,pad at break=0mm,
		before upper=\csname @totalleftmargin\endcsname0pt, % Use instead of parbox=true. This ensures parskip is inherited by box.
		% Handles box when it exists on one page only
		overlay unbroken={\draw[colexam,line width=.5pt] ([xshift=-0pt]title.north west) -- ([xshift=-0pt]frame.south west); },
		% Handles multipage box: first page
		overlay first={\draw[colexam,line width=.5pt] ([xshift=-0pt]title.north west) -- ([xshift=-0pt]frame.south west); },
		% Handles multipage box: middle page
		overlay middle={\draw[colexam,line width=.5pt] ([xshift=-0pt]frame.north west) -- ([xshift=-0pt]frame.south west); },
		% Handles multipage box: last page
		overlay last={\draw[colexam,line width=.5pt] ([xshift=-0pt]frame.north west) -- ([xshift=-0pt]frame.south west); },%
	}
	\begin{problembox}}
		{\end{problembox}\endlist}
\theoremstyle{propositionstyle}                                                                                                                               %
\newtheorem{proposition}{Proposition}[chapter]                                                                                    % Numérotation par chapitre %
\newtheorem{corollary}{Corollaire}[chapter]                                                                                                                   %
\newtheorem{theoreme}{Théorème}[chapter]                                                                                          % Numérotation par chapitre %
\newtheorem{lemma}{Lemme}[chapter]                                                                                                                            %
\newtheorem{conjecture}{Conjecture}[chapter]
\renewenvironment{proof}[1][Preuve]{                                                                                                                          %
	\par\medskip\noindent\textit{#1.}                                                                                                                         %
}{\hfill$\qed$\par\medskip}                                                                                                                                   %
\renewcommand\qedsymbol{$\blacksquare$}                                                             % Redéfinit le symbole de fin de preuve par un carré noir %
\usepackage{hyperref}                     % Ajouter des liens hypertextes dans le document (liens internes, externes, etc.), mettre les liens en surbrillance %
\usepackage{comment}                      % Permet de commenter plusieurs lignes en bloc avec \begin{comment}...\end{comment}                                 %

\usepackage{mdframed}

\setlist[itemize]{leftmargin=2em}
\setlist[enumerate]{leftmargin=2em}
